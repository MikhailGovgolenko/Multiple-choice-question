%! TEX root = fedor_main.tex

\documentclass{article}

\usepackage{geometry}
\geometry{a4paper, margin=0.6in}

\usepackage{fontspec}
\setmainfont{Times New Roman}

\usepackage{polyglossia}
\setdefaultlanguage{russian}

\usepackage{graphicx}
\usepackage{tikz}
\usepackage{pgfplots}
\usepackage{amsmath, amssymb}
\usepackage{booktabs}
\usepackage{fancyhdr}
\pagestyle{fancy}

\usepackage[colorlinks=true, linkcolor=blue, urlcolor=blue, citecolor=blue]{hyperref}
\usepackage{caption}
\usepackage{subcaption}
\usepackage{multicol, multirow}
\usepackage{wrapfig}


\def\AA{\mathring{\mathrm{A}}}

\begin{document}
	
\begin{titlepage}
	\centering
	{\scshape\LARGE Московский физико-технический институт \par}
	\vspace{3cm}
	{\scshape\Large Лабораторная работа}
	\vspace{1cm}
\line(1,0){430}\\[1mm]
	{\huge\bfseries  Иттербиевый волоконный лазер \par}
 \line(1,0){430}\\[1mm]
	\vspace{1cm}
	\vfill
\begin{flushright}
	{\large Выполнили:}\par
    {\large студенты группы Б04-305}\par
	\vspace{0.3cm}
	{\LARGE 
    
    \LARGE

    \LARGE
    
    \LARGE Халезов Федор\par
    \LARGE Владимир Рождественский\par
    \LARGE Андрей Васильев\par
    \LARGE Михаил Говголенко}
    
\end{flushright}
	

	\vfill

% Bottom of the page
	Долгопрудный, 2025 г.
\end{titlepage}
\fancyhead[L] {Волоконный лазер}
\newpage
\setcounter{page}{2}


\include{head}


\clearpage

\section{Аннотация}
В данной работе изучены генерация в волоконном лазере в режиме свободной генерации и физические основы появления релаксационных колебаний. Изучены методы создания инверсии, управления режимами генерации лазера. Оппределена зависимость мощности генерации от мощности накачки.Опредены влияние параметров генерации на частоту и затухание релаксационных колебаний.

\section{Теоретические сведения.} 
\subsection{Инверсия активной среды как необходимое условие генерации лазера}

    Излучение лазера рождается между определенными энергетическими уровнями активных центров, которые называют рабочими уровнями. Пусть $n_2$ и $n_1$ - заселенность верхнего и нижнего уровней соответственно к единице объема. Разность $N = n_2 - (g_2/g_1)n_1$ называют плотностью инверсной заселенности рабочих уровней. Считаем $g_1 = g_2$.\par
    
    Если $N > 0$, то говорят, что имеет место инверсия активной среды. В термодинамически равновесной среде величина $N$ отрицательна: заселенность верхнего уровня меньше заселенности нижнего. Для создания инверсии необходимо перевести активную среду в неравновесное состояния.\par
    
    Обеспечение инверсии активной среды является необходимой предпосылкой для реализации в лазере режима генерации. Коэффициент усиления $\chi_1$ и $\chi_2$ и коэффициент потерь описывается выражениями:
    
       \begin{equation}
            \begin{aligned}           
                &\chi_1 = \sigma_1 N,\\
                &\chi_2 = \sigma_2 N,\\
            \end{aligned}
        \end{equation}
    
    где $\sigma_1$ и $\sigma_2$ - сечение вынужденных переходов между рабочими уровнями и сечение поглощения на неактивных средах соответственно. \par
    
    Активная среда лазера характеризуется линейными коэффициентами $\chi_1$ и $\chi_2$. Распространение светового потока в активной среде хорошо описывается законов Бугера:
    
    \begin{equation}
        ds_w = [\chi_1(z) - \chi_2]S_w(z)dz
        \label{formula_Buger}
    \end{equation}
    
    Таким образом, при достаточно больших $z$ при условии $\chi_1(z) > \chi_2$, когда среда является усиливающей, возможна генерация оптического излучения.
    
\subsection{Режим свободной генерации}

    Режим свободной генерации фактически означает отсутствие какого-либо специального управления генерацией или какого-либо воздействия на нее извне. В частности, отсутствует какая-либо модуляция добротности резонатора. Свободная генерация может иметь место как в случае импульсной, так и в случае непрерывной накачки. Свободное излучение волоконного лазера представляет собой, как правило, последовательность относительно коротких импульсов, пиков. На основе одномодовой модели лазера можно показать, что регулярные затухающие пульсации связаны с переходными процессами, сопровождающими начало генерации при появлении очередного импульса накачки: иначе говоря, эти пульсации связаны с инерционностью процессов заселенности и релаксации уровней.

\subsection{Динамика генерации лазера в различных режимах лазера}

    Рассмотрим схему уровней энергии в лазере, представленную на рисунке \ref{img_energy_levels}. 


    \begin{figure}[h!]
        \centering
        \includegraphics[width=0.3\linewidth]{images/image.png}
        \caption{Энергетическая схема квази-четырехуровневого лазера}
        \label{img_energy_levels}
    \end{figure}

    
    Считая, что переходы между уровнями 4 и 3 и уровнями 2 и 1 являются быстрыми, можно положить $n_4 \approx n_2 \approx 0$. В этом случае скоростные уравнения можно записать следующим образом:
    
    \begin{equation}
        \begin{cases}
            \frac{dn_3}{dt} = W_pn_1-Bqn_3 - \frac{n_3}{\tau}\\
            \frac{dq}{dt} = V_aBqn_3 - \frac{q}{\tau_c}\\
            n_1 + n_3 = N_t,
        \end{cases}
        \label{formula_Statz_De_Mars}
    \end{equation}
    
    где $N_t$ -  полное число активных атомов в единице объема, $n_1$ -  населенность основного состояния, $n_3$ -  населенность рабочего уровня, $q$ -  полное число фотонов в резонаторе, $W_p$ -  скорость накачки, $\lambda$ -  длина волны в генераторе, $\gamma$ - потери в резонаторе за проход в одном направлении, $V_a = \frac{\pi \omega_0^2l}{4}$ - объем моды в активной среде, $B - $ скорость индуцированных переходов на один фотон в моде, $w_0$ - размер перетяжки моды в резонаторе, $L$ - длина резонатора, $l$ - длина активной среды, $L' = L +(n_0 - 1)l$ - оптическая длина резонатора, $n_0$ - показатель преломления активной среды,
    $B = \frac{\sigma lc}{V_aL'}=\frac{\sigma c}{V}, \quad \tau_c = \frac{L'}{c\gamma}$, где $V = \pi\omega_2^2L'/4$ - объем моды в резонаторе.\par
    
    Вводя инверсную населенность уровней 3 и 2 по формуле $N = n_3 - n_2 \approx n_3$ систему уравнений \ref{formula_Statz_De_Mars} можно переписать в виде
    
    
    \begin{equation}
        \begin{cases}
            \frac{dN}{dt} = W_pw_P(N_t - N) - BqN - \frac{N}{\tau}\\
            \frac{dq}{dt} = (BV_aN - \frac{1}{\tau_c})q\\
            n_1 + n_3 = N_t,
        \end{cases}
    \end{equation}
    
    Определим пороговое условие генерации лазера. Предположим, что в момент $t = 0$ в резонаторе, вследствие спонтанного испускания, присутствует некоторое небольшое число фотонов $q$. При этом из уравнения следует, что для того, чтобы величина $dq/dt$ была положительной, должно выполняться условие $V_aBN - 1/\tau_c) > 0$. В этом случае генерация возникает, если инверсия населенностей $N$ достигает некоторого критического значения $N_c$, определяемого выражением:
    
    \begin{equation}
        N_c = \frac{1}{V_aB\tau_c} = \frac{\tau}{\sigma l},
    \end{equation}
    
    где $\sigma$ - сечение перехода генерации, $\gamma$ - суммарные потери в резонаторе за проход в одном направлении.\par
    
    Таким образом, критическая скорость накачки соответствует ситуации, когда полная скорость накачки уровней уравновешивает скорость спонтанных переходов с рабочего уровня:
    
    \begin{equation}
        W_{cp} = \frac{N_c}{(N_t - N_c)\tau} = \frac{1}{\tau(\tau_cV_aBN_t - 1)}
    \end{equation}
    
    Если $W_p > W_cp$, то число фотонов $q$ будет возрастать от начального значения, определяемого спонтанным изучением, и если $W_p$ не зависит от времени, то, в конце концов, достигнет некоторого постоянного значения $q_0$. Это стационарное значение и соответствующее ему стационарное значение инверсии $N_0$ получают из уравнения (\ref{formula_Statz_De_Mars}), если в них положить $\dot{N} = \dot{q} = 0$
    
    \begin{equation}
        \begin{aligned}
            &N_0 = \frac{1}{V_aB\tau_c} = N_c, \\
            &q_0 = V_a\tau_c \left[ W_p(N_t - N_0) - \frac{N_0}{\tau} \right]
        \end{aligned}
    \end{equation}
    
    Полученные уравнения описывают непрерывный режим работы четырехуровневого лазера. 
    
\subsection{Релаксационные колебания в лазере. Работа лазера в нестационарных режимах генерации при ступенчатом включении импульса накачки}

    Рассмотрим работу лазера при нестационарной накачке. В случае, когда скорость накачки описывается ступенчатой функцией, будем считать, что $W_p = 0$,  $t<0$ и $W_p(t) = W_p$, $t>0$. При небольших колебаниях инверсии и количества фотонов около стационарных значений $N_0$ и $q_0$, можно записать
    \begin{equation}
        \begin{cases}
            N(t) = N_0 + \delta N\\
            q(t) = q_0 + \delta q
        \end{cases}
    \end{equation}
    
    Тогда получаем систему уравнений 
    
    \begin{equation}
        \begin{cases}
            \dot{\delta N} = -\delta N(W_p + \frac{1}{\tau_{уровня}}) - B(q_0 \delta N + N_0\delta q)\\
            \dot{\delta q} = Bq_0V_a\delta N
        \end{cases}
    \end{equation}
    
    
    Подстановка второго уравнения в первое с учетом $BV_a N - \frac{1}{\tau_{фотона}} = 0$ дает квадратное уравнения. Решая его, получаем 
    
    \begin{equation}
        \delta q = \delta q_0\cdot exp(st),
    \end{equation}
    
    где $s = -1/t_0 \pm[(1/t_0)^2 - w^2]^{1/2}$. Для случая $1/t_0 < w$ получаем:
    
    \begin{equation*}
        s = -(1/t_0) \pm iw', \quad \textit{где } w' = [w^2 - (1/\tau_{фотона})^2]^{1/2}
    \end{equation*}
    
    После преобразований можно получить формулы:

    \begin{equation}
        t_0 = 2\tau_\text{уровня}/x
        \label{formula_t_0}
    \end{equation}
    
    \begin{equation}
        \omega = \sqrt{\frac{x - 1}{\tau_{фотона}\tau_\text{уровня}}}
        \label{formula_omega(x)}
    \end{equation}
    
     где $x = \frac{W_p}{W_{cp}}$ - превышение скорости накачки над пороговой, $t_0$ - время затухания, $w'$ - период релаксационных колебаний. Таким образом, при ступенчатом включении накачки при генерации лазера происходят затухающие релаксационные колебания количества фотонов в резонаторе и, следовательно, выходной мощности с частотой $w'$.
     
\section{Установка}

\begin{figure}[h!]
    \centering
    \includegraphics[width=0.8\linewidth]{images/Screenshot 2025-11-23 213713.png}
    \caption{Схема установки}
    \label{fig:placeholder}
\end{figure}

\clearpage

\section{Ход работы}

\subsection{Без модуляции точка накачки}

Определим зависимость мощности генерации от мощности накачки. Мощность накачки \(W_{\text{нак}} =  IV\), где \(I\) изменяемо нами для получения зависимости, а \(V = 6\)В задается источником питания.

Также заметим что при \(I \approx 0.912\) А (соответсвенно \(W_{\text{нак}} \approx 5.5\) Вт) генерация перестает наблюдаться. Это порог генерации лазера.

\begin{table}[h!]
    \centering
    \begin{tabular}{|c|c|c|c|c|c|}
    \hline
         I, A& \(W_{\text{нак}}\), \(\text{Вт}\) & \(P_{\text{ген}}\), \(\text{Дж}\)\\\hline \hline
         %0.977 & 
         1.2&  7.2& 7.5\\\hline
         1.4&  8.4&13.5\\\hline
         1.8&  9.6 &26\\\hline
         2.2&  13.2 &38\\\hline
         2.6&  15.6 &50\\\hline
         3.0&  18.0& 70\\\hline
         %
    \end{tabular}
    \caption{Зависимость мощности генерации от мощности накачки}
    \label{tab:placeholder}
\end{table}

ИПТ-3 показывает значение мощности в Джоулях (в связи с этим используем другую букву для обозначения этого значения "мощности"). Это значение "мощности" явно пропорционально действительному значению мощности в Ваттах, однако в результате оценка КПД не представляется возможной. 

Построим график зависимости мощности излучения от мощности накачки:

\begin{figure}[h!]
    \centering
    \includegraphics[width=0.9\linewidth]{images/cpd_almost.png}
    \caption{Зависимости мощности излучения от мощности накачки}
    \label{fig:placeholder}
\end{figure}


Из графика находим что  \(P_{\text{ген}} = (5.4 \pm 0.4)W_{\text{нак}} - 30.8\pm 5.0\) Дж.

При пороговой мощности генерации \(P_{\text{ген}} = 0\), откуда \(W_{\text{пор}} = 5.7\pm 1.0\)Вт, что хорошо согласуется со знаением, полученным "на глаз".


\clearpage

\subsection{С модуляцией точка накачки}

\subsubsection{Вид релаксационных колебаний}

\begin{figure}[h!]
    \centering
    \includegraphics[width=0.75\linewidth]{images/relax_better.png}
    \caption{Релаксационные колебания}
    \label{fig:placeholder}
\end{figure}

\begin{figure}[h!]
    \centering
    \includegraphics[width=0.75\linewidth]{images/relax_2.png}
    \caption{Релаксационные колебания}
    \label{fig:placeholder}
\end{figure}

Отсюда видно что время между между пульсами с генератора можно оценить как \(\approx 150\)нс. Откуда, зная что частота импульсов сотсавляет 1000Гц, получаем что скважность равна \(k = 85\%\). 

\subsubsection{Частота релаксационных колебаний}

Получим зависимость частоты релаксационных колебаний от мощности накачки.

Значение частоты релаксационных колебаний находилось как среднее значение частот первых n колебаний.

Мощность накачки находим с учетом скважности как \(W_{\text{нак}} = kIV\).


\begin{table}[h!]
    \centering
    \begin{tabular}{|c|c|c|}
    \hline
        I, A & \(W_{\text{нак}}\), Вт & \(\nu\), кГц \\
\hline
1.1 & 5.6 & 25.47 \\
\hline
1.2 & 6.1 & 32.73 \\
\hline
1.3 & 6.6 & 27.69 \\
\hline
1.4 & 7.1 & 32.79 \\
\hline
1.5 & 7.6 & 37.24 \\
\hline
1.6 & 8.2 & 40.2 \\
\hline
1.7 & 8.7 & 39.70 \\
\hline
1.8 & 9.2 & 46.14 \\
\hline
2.0 & 10.2 & 49.55 \\
\hline
2.2 & 11.2 & 53.38 \\
\hline
2.6 & 13.3 & 61.76 \\
\hline
2.875 & 14.7 & 66.67 \\
\hline
         
         
         
         %\cline{1-1}
    \end{tabular}
    \caption{Зависимость релаксационных колебаний от мощности накачки}
    \label{tab:placeholder}
\end{table}

 

\subsubsection{Зависимость от мощности накачки}

С учетом порога генерации \(W_{\text{пор}} = 5.5\)Вт, построю график частоты генерации от превышения мощности над мощностью начала генерации.

\begin{figure}[h!]
    \centering
    \includegraphics[width=0.9\linewidth]{images/relaxation_plot.png}
    \caption{Зависимость частоты релаксационных колебаний от превышения над порогом генерации}
    \label{fig:placeholder}
\end{figure}







\clearpage


\subsection{Распределение по частоте излучения лазера}

\subsubsection{Наблюдаемые значения}

\begin{figure}[h!]
    \centering
    \includegraphics[width=0.75\linewidth]{images/spector120.png}
    \caption{Пример спектора излучения лазера в диапазоне частот 0-120кГц}
    \label{fig:placeholder}
\end{figure}

Заметим что возможно наблюдать вторую моду генерации
%я не записал частоту на фотке, взял и таблицы
\begin{figure}[h!]
    \centering
    \includegraphics[width=0.75\linewidth]{images/2mode.png}
    \caption{I = 1.8А, диапазон 0-25МГц}
    \label{fig:placeholder}
\end{figure}

Частоты мод:

\begin{table}[h!]
    \centering
    \begin{tabular}{|c|c|c|c|c|}
    \hline
        I, A & \(\nu_1\), МГц & \(\sigma_{\nu_1}\), МГц & \(\nu_2\), МГц & \(\sigma_{\nu_2}\), МГц\\\hline
        \hline
        1.8 & 7.55 & 0.71 & 18.19 & 0.22\\\hline
    \end{tabular}
    \caption{Моды генерации}
    \label{tab:placeholder}
\end{table}

Здесь погрешность значния пика равна его полуширине
\newpage
\subsubsection{Определение добротности}

Будем рассматривать собственную моду генерации лазера. Снова погрешность значния пика равна его полуширине. В этом случае получаем что \(Q = \frac{\nu}{\sigma_{\nu}}\)

\begin{table}[h!]
    \centering
    \begin{tabular}{|c|c|c|c|}
    \hline
        I, A & \(\nu\), кГц & \(\sigma_{\nu}\), кГц & Q\\\hline
         1.6  & 40.71   & 1.31    & 31.08 \\
\hline
1.8  & 45.405  & 3.205   & 14.17  \\
\hline
2.2  & 55.08   & 4.76    & 11.57  \\
\hline
2.6  & 60.515  & 4.215   & 14.36  \\
\hline
1.4  & 31.525  & 4.155   & 7.59  \\
\hline
1.2  & 24.08   & 4.76    & 5.06  \\
\hline
    \end{tabular}
    \caption{Значнеие собственной моды лазера}
    \label{tab:placeholder}
\end{table}

\subsubsection{Характерное время затухания $t_0$ и время жизни рабочем уровне $\tau$ для Yb$^{3+}$}
Cм. таблицу \ref{table_lasers_comparisson}

\section{Задачи}

\hspace{1.5em}\textbf{1.} Какие особенности в построении оптической схемы для волоконного лазера по сравнению с твердотельным лазером?

\vspace{0.5em}
\textbf{Ответ.} Брегговские решётки, выполняющие в волоконном лазере ту же функцию, что и зеркала в твердотельном, являются неотъемлемой частью самого волокна. Благодаря этому система получается монолитной и не требует юстировки.

Большое отношение площади поверхности волокна к его объёму позволяет эффективно отводить тепло с помощью простого радиатора, даже при высоких мощностях накачки. В твердотельном лазере рабочий стержень при этом сильно нагревается.

Для выделения конкретных мод излучения твердотельного лазера необходимы внутрирезонаторные диафрагмы, что приводит к потерям мощности. Волоконный лазер формирует излучение конкретной моды благодаря волновым свойствам самого волокна. Кроме того, волоконный резонатор можно изгибать, что значительно уменьшает размеры лазера.

\vspace{1em}
\textbf{2.} Каковы методы создания и особенности работы брегговских отражающих элементов в волоконном лазере?

\vspace{0.5em}
\textbf{Ответ.} Для реализации режима генерации необходима положительная обратная связь. В волоконном лазере она обеспечивается отражением излучения на брегговских решётках, встроенных на концах волокна. Электромагнитная волна, распространяющаяся по волоконному волноводу, многократно отражается от этих решёток, усиливаясь при каждом прохождении 
через активное волокно.

Если одну из брегговских решёток сделать частично пропускающей, на выходе лазера формируется пучок полезного излучения.

\vspace{1em}
\textbf{3.} Как осуществляется накачка в волоконном лазере?

\vspace{0.5em}
\textbf{Ответ.} В волоконных лазерах активное волокно состоит из трёх основных частей: сердцевины, легированной ионами редкоземельных металлов; внутренней оболочки, образующей с сердцевиной волновод; и внешней оболочки, по которой распространяется 
излучение накачки, вводимое от полупроводникового лазера.

Для излучения накачки волновод является многомодовым, в то же время сердцевина активной области формирует одномодовый волновод для генерируемого излучения.

Для ввода излучения накачки используют несколько методов:
\begin{enumerate}
    \item Торцевой ввод.
    \item Набор V-образных канавок, распределённых по боковой поверхности световода.
    \item Два светодиода, размещаемых в общей оболочке: один из них активный, а другой 
    вводит излучение накачки, которое вместе с их контактом проходит в активную область 
    и осуществляет накачку.
\end{enumerate}

Таким образом, реализуется распределённая накачка активной области.


\vspace{1em}
\textbf{4.} С чем связана нерегулярность пичков выходного излучения лазера?

\vspace{0.5em}
\textbf{Ответ.} Природа пиков до сих пор остаётся предметом исследований. На основе одномодовой модели лазера можно показать, что регулярные затухающие пульсации связаны с переходными процессами, возникающими при начале генерации и появлении очередного импульса накачки. Иными словами, эти пульсации обусловлены инерционностью процессов заселения и релаксации уровней. Существенное влияние на характер пикового режима оказывает многомодовость генерации. В частности, наличие большого числа мод может вносить в картину пульсаций неупорядоченность.

\vspace{1em}
\textbf{5.} Рассчитать пороговую и стационарную инверсию в лазере.

\vspace{0.5em}
\textbf{Решение.}
\[N_c=N_0=\frac{1}{V_a B\tau_c}=\frac{\gamma}{\sigma l}=\frac{0.12}{2.5\cdot10^{-20}\cdot 500}\,\text{см}^{-3}=9.6\cdot 10^{15}\,\text{см}^{-3},\]
Где $N_c$ и $N_0$ --- пороговая и стационарная инверсии соответсвенно.

\vspace{1em}
\textbf{6.} Какова природа релаксационных колебаний в лазерах. Чем определяется их характерная частота колебаний.

\vspace{0.5em}
\textbf{Ответ.} Под действием накачки увеличивается сте-
пень инверсионной населённости атомов активного вещества. При достижении порогового значения инверсионной населённости атомы активного вещества начинают излучать большое количество фотонов, благодаря чему мощность излучения резко возрастает. Система «проскакивает» положение равновесия и инверсионная населённость резко уменьшается, так как мощность накачки становится гораздо меньше генерируемой мощности. Когда инверсионная населённость становится ниже порогового уровня, число фотонов в резонаторе резко уменьшается и снова происходит накачка системы. Цикл повторяется с уменьшением амплитуды, так как система стремится к состоянию динамического равновесия (это так же показывают уравнения колебаний).

\[\omega = \sqrt{\frac{x-1}{\tau_0\tau}}\]

\vspace{1em}
\textbf{7.} Чем определяется длительность импульсов при релаксационных колебаниях.

\vspace{0.5em}
\textbf{Ответ.}
\[t_0=\frac{2\tau}{x}\]

\vspace{1em}
\textbf{8.} Рассчитать частоту релаксационных колебаний для волоконного лазера, используемого в работе.

\vspace{0.5em}
\textbf{Решение.} По формуле (5.22) из лаборатрного практикума
\[\omega=\sqrt{\frac{x-1}{\tau_0 \tau}},\]
где $x=\frac{W_p}{W_{cp}}$, $\frac{1}{\tau_0}=-\frac{c}{2L}\ln(1-T)$, $T=0.8$, $\tau=1200\,\text{мкс}$, $L=10\,\text{см}$. Для $x=1.5$,
\[\omega=\sqrt{\frac{1-\frac{W_p}{W_{cp}}}{\tau}\frac{c}{2L}\ln(1-T)}\approx 1002945\,\text{Гц}\approx 1\,\text{МГц}.\]

\vspace{1em}
\textbf{9.} Определить время затухания фотонов в резонаторе волоконного лазера.

\vspace{0.5em}
\textbf{Решение.}
\[\tau_0=\frac{-2L}{c\ln(1-T)}\approx150\,\text{нс}.\]

\vspace{1em}
\textbf{10.} Вычислить частоту межмодового интервала для продольных мод лазера, используемого в работе.

\vspace{0.5em}
\textbf{Решение.}
\[\Delta\nu=\frac{c}{2L}=30\,\text{МГц},\]
где $L=10\,\text{м}$

\vspace{1em}
\textbf{11.} Определить температуру, при которой работа иттербиевого лазера будет происходить по трёхуровневой схеме, если штарковское расщепление уровней рабочего перехода $\sim 500\,\text{см}^{-1}$.

\vspace{0.5em}
\textbf{Решение.} Пусть штарковское расщепление уровней рабочего перехода $\xi \sim 500\,\text{см}^{-1}$. Тогда 
\[E=\hbar \omega=\hbar c\frac{2\pi}{\lambda}\sim k_\text{Б},\]
\[T=\frac{2\pi \hbar c}{k_\text{Б}}=\frac{hc}{\lambda k_\text{Б}}=\frac{hc}{k_\text{Б}}\xi=720\,\text{K}.\]

\vspace{1em}
\textbf{12.} Рассчитать частоту и время затухания релаксационных колебаний для типичного He-Ne, полупроводникового и YAG:Nd$^{3+}$ лазеров. Сделать выводы о возможности наблюдения релаксационных колебаний в этих лазерах.

\vspace{0.5em}
\textbf{Решение.}
\[t_0=\frac{2\tau}{x}\]
\[\omega=\sqrt{\frac{x-1}{\tau_0\tau}}=\sqrt{\frac{1-x}{\tau}\frac{c}{2L}\ln(1-T)}\]
\[N=\frac{\omega t_0}{2\pi}\]

\begin{table}[h!]
\centering
\begin{tabular}{|c|c|c|c|c|c|c|}
    \hline
    тип лазера & $L$, м & $T$ & $\tau$ & $t_0$ & $\omega$, рад/с & $N$ \\ \hline
    He-Ne & 0.5 & 0.99 & 100 нс & 133 нс & $83.1\cdot 10^6$ & 1.8 \\ \hline
    YAG:Nd$^{3+}$ & 0.5 & 0.97 & 230 мкс & 307 мкс & $1.5\cdot 10^6$ & 73.3 \\ \hline
    полупров. & $5\cdot 10^{-4}$ & 0.5 & 1 нс & 1.3 нс & $10.2\cdot10^9$ & 2.1 \\ \hline
    волоконный & 5 & 0.2 & 1400 мкс & 1.87 мс & $48.8\cdot 10^3$ & 14.5 \\ \hline
\end{tabular}
\caption{Сравнение характеристик различных лазеров. $x=1.5.$}
\label{table_lasers_comparisson}
\end{table}

\textbf{He-Ne:} частота релаксационных колебаний сравнительно невысока, при этом $N \approx 1.8$, то есть до затухания происходит всего 1--2 колебания. Во временной области наблюдается один или два переходных всплеска, а устойчивые многопериодные релаксационные колебания практически не формируются.

\textbf{YAG:Nd$^{3+}$:} низкая частота и большое число циклов --- релаксационные колебания чётко наблюдаемы во временной области на обычной аппаратуре.

\textbf{Полупроводниковый:} частота очень высокая и $N\approx 2$. Во временной области нужны фотодиод и высокоскоростной осциллограф/электроника --- только тогда можно увидеть 1--2 колебания.





\end{document}
