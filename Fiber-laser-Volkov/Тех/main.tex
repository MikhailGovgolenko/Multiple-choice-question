\documentclass[a4paper, 12pt]{article}
\usepackage[a4paper,top=1.5cm, bottom=1.5cm, left=1cm, right=1cm]{geometry}
\usepackage[utf8]{inputenc}
\usepackage{mathtext}
\usepackage{amsmath}
\usepackage{amsfonts}
\usepackage[english, russian]{babel}
\usepackage{indentfirst}
\usepackage{longtable}
\usepackage{graphicx}
\graphicspath{{images/}}
\DeclareGraphicsExtensions{.pdf,.png,.jpg}
\usepackage{natbib}

\usepackage{float}  % принудительное размещение таблицы ко кодовой букве H вместо !ht
\restylefloat{table}

\usepackage{letltxmacro}  % новый дизайн корня(с закорючкой вниз на конце)
\makeatletter
\let\oldr@@t\r@@t
\def\r@@t#1#2{
\setbox0=\hbox{$\oldr@@t#1{#2\,}$}\dimen0=\ht0
\advance\dimen0-0.2\ht0
\setbox2=\hbox{\vrule height\ht0 depth -\dimen0}
{\box0\lower0.4pt\box2}}
\LetLtxMacro{\oldsqrt}{\sqrt}
\renewcommand*{\sqrt}[2][\ ]{\oldsqrt[#1]{#2} }
\makeatother

\usepackage{wrapfig}  % рисунки с обтеканием

\usepackage{color,soul}  % новая команда \ul - подчеркивание красной линией
\setulcolor{red} 

\usepackage{breqn} % пак для создания двустрочных equation через \begin{dmath}

\usepackage{nicematrix}  % цветные таблицы

\usepackage{siunitx}  % единицы СИ

\usepackage{subcaption} % Пакет для подфигур

\renewcommand\thesubfigure{\asbuk{subfigure}}  % индексы в subfigure на русском

\usepackage[table,xcdraw]{xcolor}  % цветные таблицы


\newcommand{\upperRomannumeral}[1]{\uppercase\expandafter{\romannumeral#1}}
\begin{document}
\begin{titlepage}
    \newpage
    \begin{center}
     Министерство науки и высшего образования Российской федерации \\ ФЕДЕРАЛЬНОЕ ГОСУДАРСТВЕННОЕ АВТОНОМНОЕ \\ ОБРАЗОВАТЕЛЬНОЕ УЧРЕЖДЕНИЕ ВЫСШЕГО ОБРАЗОВАНИЯ \\ «МОСКОВСКИЙ ФИЗИКО-ТЕХНИЧЕСКИЙ ИНСТИТУТ \\ (НАЦИОНАЛЬНЫЙ ИССЛЕДОВАТЕЛЬСКИЙ УНИВЕРСИТЕТ)» \\ (МФТИ)
    \end{center}
    
    \vspace{15em}
    
    \begin{center}
    КАФЕДРА КВАНТОВОЙ ЭЛЕКТРОНИКИ \\
    \vspace{1em}
    ОТЧЕТ\\
    ПО ЛАБОРАТОРНОЙ РАБОТЕ \\
    \vspace{1em}
   ВОЛОКОННЫЙ ЛАЗЕР
    \end{center}

    \vspace{7em}
    \begin{flushleft}
        Работу выполнили \hspace{17em} \underline{\hspace{3cm}}
        К.С. Колинько \\
        \hspace{26em} \underline{\hspace{3cm}} Р.В. Волков\\
        \hspace{26em} \underline{\hspace{3cm}} Н.А. Соколов\\
        \hspace{26em} 
        \raisebox{-\baselineskip}{\shortstack{\underline{\hspace{3cm}}\\(подпись, дата)}}    К.А. Виноградов\\
       
    \end{flushleft}

    \vspace{1em}

    \begin{flushleft}
        Работу принял, оценка
        \hspace{15em}
        \raisebox{-\baselineskip}{\shortstack{\underline{\hspace{5cm}}\\(подпись, дата, оценка)}}
    \end{flushleft}

    \vspace{5em}
    
    \begin{center}
        Долгопрудный, 2024
    \end{center}
\end{titlepage}

\newpage
\tableofcontents

\newpage
\section{Цели работы}
    \begin{enumerate}
        \item Изучить генерацию в волоконном лазере в режиме свободной генерации и физические основы появления релаксационных колебаний.
        \item Изучить методы создания инверсии, управления режимами генерации лазера.
        \item Определить влияние параметров генерации на частоту и затухание релаксационных колебаний.
    \end{enumerate}

\section{Теоретические сведения.} 
\subsection{Инверсия активной среды как необходимое условие генерации лазера}

    Излучение лазера рождается между определенными энергетическими уровнями активных центров, которые называют рабочими уровнями. Пусть $n_2$ и $n_1$ - заселенность верхнего и нижнего уровней соответственно к единице объема. Разность $N = n_2 - (g_2/g_1)n_1$ называют плотностью инверсной заселенности рабочих уровней. Считаем $g_1 = g_2$.\par
    
    Если $N > 0$, то говорят, что имеет место инверсия активной среды. В термодинамически равновесной среде величина $N$ отрицательна: заселенность верхнего уровня меньше заселенности нижнего. Для создания инверсии необходимо перевести активную среду в неравновесное состояния.\par
    
    Обеспечение инверсии активной среды является необходимой предпосылкой для реализации в лазере режима генерации. Коэффициент усиления $\chi_1$ и $\chi_2$ и коэффициент потерь описывается выражениями:
    
       \begin{equation}
            \begin{aligned}           
                &\chi_1 = \sigma_1 N,\\
                &\chi_2 = \sigma_2 N,\\
            \end{aligned}
        \end{equation}
    
    где $\sigma_1$ и $\sigma_2$ - сечение вынужденных переходов между рабочими уровнями и сечение поглощения на неактивных средах соответственно. \par
    
    Активная среда лазера характеризуется линейными коэффициентами $\chi_1$ и $\chi_2$. Распространение светового потока в активной среде хорошо описывается законов Бугера:
    
    \begin{equation}
        ds_w = [\chi_1(z) - \chi_2]S_w(z)dz
        \label{formula_Buger}
    \end{equation}
    
    Таким образом, при достаточно больших $z$ при условии $\chi_1(z) > \chi_2$, когда среда является усиливающей, возможна генерация оптического излучения.
    
\subsection{Режим свободной генерации}

    Режим свободной генерации фактически означает отсутствие какого-либо специального управления генерацией или какого-либо воздействия на нее извне. В частности, отсутствует какая-либо модуляция добротности резонатора. Свободная генерация может иметь место как в случае импульсной, так и в случае непрерывной накачки. Свободное излучение волоконного лазера представляет собой, как правило, последовательность относительно коротких импульсов, пиков. На основе одномодовой модели лазера можно показать, что регулярные затухающие пульсации связаны с переходными процессами, сопровождающими начало генерации при появлении очередного импульса накачки: иначе говоря, эти пульсации связаны с инерционностью процессов заселенности и релаксации уровней.

\subsection{Динамика генерации лазера в различных режимах лазера}

    Рассмотрим схему уровней энергии в лазере, представленную на рисунке \ref{img_energy_levels}. 
    
    \begin{figure}[H]
        \centering
        \includegraphics[width=5cm]{images/energy_levels.png}
        \caption{Энергетическая схема квази-четырехуровневого лазера}
        \label{img_energy_levels}
    \end{figure}
    
    Считая, что переходы между уровнями 4 и 3 и уровнями 2 и 1 являются быстрыми, можно положить $n_4 \approx n_2 \approx 0$. В этом случае скоростные уравнения можно записать следующим образом:
    
    \begin{equation}
        \begin{cases}
            \frac{dn_3}{dt} = W_pn_1-Bqn_3 - \frac{n_3}{\tau}\\
            \frac{dq}{dt} = V_aBqn_3 - \frac{q}{\tau_c}\\
            n_1 + n_3 = N_t,
        \end{cases}
        \label{formula_Statz_De_Mars}
    \end{equation}
    
    где $N_t$ -  полное число активных атомов в единице объема, $n_1$ -  населенность основного состояния, $n_3$ -  населенность рабочего уровня, $q$ -  полное число фотонов в резонаторе, $W_p$ -  скорость накачки, $\lambda$ -  длина волны в генераторе, $\gamma$ - потери в резонаторе за проход в одном направлении, $V_a = \frac{\pi \omega_0^2l}{4}$ - объем моды в активной среде, $B - $ скорость индуцированных переходов на один фотон в моде, $w_0$ - размер перетяжки моды в резонаторе, $L$ - длина резонатора, $l$ - длина активной среды, $L' = L +(n_0 - 1)l$ - оптическая длина резонатора, $n_0$ - показатель преломления активной среды,
    $B = \frac{\sigma lc}{V_aL'}=\frac{\sigma c}{V}, \quad \tau_c = \frac{L'}{c\gamma}$, где $V = \pi\omega_2^2L'/4$ - объем моды в резонаторе.\par
    
    Вводя инверсную населенность уровней 3 и 2 по формуле $N = n_3 - n_2 \approx n_3$ систему уравнений \ref{formula_Statz_De_Mars} можно переписать в виде
    
    
    \begin{equation}
        \begin{cases}
            \frac{dN}{dt} = W_pw_P(N_t - N) - BqN - \frac{N}{\tau}\\
            \frac{dq}{dt} = (BV_aN - \frac{1}{\tau_c})q\\
            n_1 + n_3 = N_t,
        \end{cases}
    \end{equation}
    
    Определим пороговое условие генерации лазера. Предположим, что в момент $t = 0$ в резонаторе, вследствие спонтанного испускания, присутствует некоторое небольшое число фотонов $q$. При этом из уравнения следует, что для того, чтобы величина $dq/dt$ была положительной, должно выполняться условие $V_aBN - 1/\tau_c) > 0$. В этом случае генерация возникает, если инверсия населенностей $N$ достигает некоторого критического значения $N_c$, определяемого выражением:
    
    \begin{equation}
        N_c = \frac{1}{V_aB\tau_c} = \frac{\tau}{\sigma l},
    \end{equation}
    
    где $\sigma$ - сечение перехода генерации, $\gamma$ - суммарные потери в резонаторе за проход в одном направлении.\par
    
    Таким образом, критическая скорость накачки соответствует ситуации, когда полная скорость накачки уровней уравновешивает скорость спонтанных переходов с рабочего уровня:
    
    \begin{equation}
        W_{cp} = \frac{N_c}{(N_t - N_c)\tau} = \frac{1}{\tau(\tau_cV_aBN_t - 1)}
    \end{equation}
    
    Если $W_p > W_cp$, то число фотонов $q$ будет возрастать от начального значения, определяемого спонтанным изучением, и если $W_p$ не зависит от времени, то, в конце концов, достигнет некоторого постоянного значения $q_0$. Это стационарное значение и соответствующее ему стационарное значение инверсии $N_0$ получают из уравнения (\ref{formula_Statz_De_Mars}), если в них положить $\dot{N} = \dot{q} = 0$
    
    \begin{equation}
        \begin{aligned}
            &N_0 = \frac{1}{V_aB\tau_c} = N_c, \\
            &q_0 = V_a\tau_c \left[ W_p(N_t - N_0) - \frac{N_0}{\tau} \right]
        \end{aligned}
    \end{equation}
    
    Полученные уравнения описывают непрерывный режим работы четырехуровневого лазера. 
    
\subsection{Релаксационные колебания в лазере. Работа лазера в нестационарных режимах генерации при ступенчатом включении импульса накачки}

    Рассмотрим работу лазера при нестационарной накачке. В случае, когда скорость накачки описывается ступенчатой функцией, будем считать, что $W_p = 0$,  $t<0$ и $W_p(t) = W_p$, $t>0$. При небольших колебаниях инверсии и количества фотонов около стационарных значений $N_0$ и $q_0$, можно записать
    \begin{equation}
        \begin{cases}
            N(t) = N_0 + \delta N\\
            q(t) = q_0 + \delta q
        \end{cases}
    \end{equation}
    
    Тогда получаем систему уравнений 
    
    \begin{equation}
        \begin{cases}
            \dot{\delta N} = -\delta N(W_p + \frac{1}{\tau_{уровня}}) - B(q_0 \delta N + N_0\delta q)\\
            \dot{\delta q} = Bq_0V_a\delta N
        \end{cases}
    \end{equation}
    
    
    Подстановка второго уравнения в первое с учетом $BV_a N - \frac{1}{\tau_{фотона}} = 0$ дает квадратное уравнения. Решая его, получаем 
    
    \begin{equation}
        \delta q = \delta q_0\cdot exp(st),
    \end{equation}
    
    где $s = -1/t_0 \pm[(1/t_0)^2 - w^2]^{1/2}$. Для случая $1/t_0 < w$ получаем:
    
    \begin{equation*}
        s = -(1/t_0) \pm iw', \quad \textit{где } w' = [w^2 - (1/\tau_{фотона})^2]^{1/2}
    \end{equation*}
    
    После преобразований можно получить формулы:

    \begin{equation}
        t_0 = 2\tau_{уровня}/x
        \label{formula_t_0}
    \end{equation}
    
    \begin{equation}
        \omega = \sqrt{\frac{x - 1}{\tau_{фотона}\tau_{уровня}}}
        \label{formula_omega(x)}
    \end{equation}
    
     где $x = \frac{W_p}{W_{cp}}$ - превышение скорости накачки над пороговой, $t_0$ - время затухания, $w'$ - период релаксационных колебаний. Таким образом, при ступенчатом включении накачки при генерации лазера происходят затухающие релаксационные колебания количества фотонов в резонаторе и, следовательно, выходной мощности с частотой $w'$.
     
\section{Результаты измерений и обработка данных.}
\subsection{Измерение зависимости мощности лазера от тока накачки.}
    Измерим величину энергии лазерного излучения $W_{излуч}$ при различных значениях тока накачки $I$. Результаты представлены в таблице \ref{table_W(I)}.\par
    
    Напряжение накачки в течение всех измерений не менялось и было равно $U = 6$ В. Поэтому $P_{накач} = U \cdot I$. Перевод энергии излучения лазера $W_{излуч}$ в мощность излучения $P_{излуч}$ осуществляется через линейный коэффициент перевода $P_{излуч} = \kappa W_{излуч}$, где $\kappa = (270\pm 1,8)\cdot 10^{-4}\ \frac{Вт}{Дж} = (27\pm 0,18)\ \frac{мВт}{Дж}$.

    \begin{table}[H]
    \centering
    \begin{tabular}{|c|c|c|c|c|c|c|c|c|c|c|c|c|c|c|}
    \hline
        № & 1 & 2 & 3 & 4 & 5 & 6 & 7 & 8 & 9 & 10 & 11 & 12 & 13 & 14  \\ \hline
        $W_{излуч}$, Дж & 1,8 & 4,5 & 7,2 & 9,6 & 12,3 & 17 & 20 & 23 & 27 & 30 & 34 & 36 & 39 & 42 \\ \hline
        $I$, А & 0,98 & 1,11 & 1,22 & 1,31 & 1,42 & 1,53 & 1,6 & 1,72 & 1,82 & 1,92 & 2,05 & 2,13 & 2,21 & 2,36 \\ \hline
    \end{tabular}
    \caption{Результаты измерений $P$ при различных значениях $I$.}
    \label{table_W(I)}
    \end{table}

    На рисунке \ref{img_P(P)_for_P} представлен график зависимости $P_{излуч}(I)$, построенный по  данным таблицы \ref{table_W(I)}. По первым пяти точкам методом \textit{scipy.optimize.curve_fit} проведена интерполяция прямой вида $y = k(x - x_0)$. Точка пересечения этой прямой с осью абсцисс соответствует пороговой мощности накачки $P_{порог}$, при котором в лазере начинается генерация. Соответственно, $\boxed{P_{порог} = (5,418 \pm 0,004)\ Вт}$. Отметим на будущее, что значение порогового тока $I_{порог} = (0,91 \pm 0,01)\ Вт$.\par

    На рисунке \ref{img_P(P)_for_eta} представлен график зависимости мощности лазерного излучения $P_{излуч}$ от мощности накачки $P_{накач}$. Интерполяция проводилась МНК по всем экспериментальным точкам. Тогда КПД лазера определяется угловым коэффициентом $k = \frac{P_{излуч}}{P_{накач}}$. Тогда из рисунка \ref{img_P(P)_for_eta}:
    
    $$\boxed{\eta = (124.5\pm 0.3)\ \frac{мВт}{Вт} = (0,1245\pm 0,0003) = 12,45\%}$$

    \begin{figure}[H]
        \begin{subfigure}{0.5\linewidth} 
            \includegraphics[width=\linewidth]{images/P(P)_for_P.png}
            \caption{}
            \label{img_P(P)_for_P}
        \end{subfigure}
        \hfill
        \begin{subfigure}{0.5\linewidth} 
            \includegraphics[width=\linewidth]{images/P(P)_for_eta.png}
            \caption{}
            \label{img_P(P)_for_eta}
        \end{subfigure}
        \caption{Графики зависимости $P_{излуч}(P_{накач})$: \ref{img_P(P)_for_P} -- интерполяция по первым пяти точкам, \ref{img_P(P)_for_eta} -- интерполяция по всем точкам.}
        \label{img_two_P(P)}
    \end{figure}

\subsection{Исследование релаксационных колебаний.}
    Включим генератор импульсов и будем создавать импульсы тока накачки $I = I_0 \pm \delta I$. Количество фотонов в резонаторе будет при этом испытывать релаксационные колебания. Рассмотрим осциллограммы сигналов релаксационных колебаний при различных значениях $I_0$ -- рисунок \ref{img_peaks}.

    \begin{figure}[H]
        \begin{subfigure}{0.5\linewidth} 
            \includegraphics[width=\linewidth]{images/peaks_1,19.png}
            \caption{}
            \label{img_peaks_1,19}
        \end{subfigure}
        \hfill
        \begin{subfigure}{0.5\linewidth} 
            \includegraphics[width=\linewidth]{images/peaks_1,49.png}
            \caption{}
            \label{img_peaks_1,49}
        \end{subfigure}
        \vfill
        \begin{subfigure}{0.5\linewidth} 
            \includegraphics[width=\linewidth]{images/peaks_1,76.png}
            \caption{}
            \label{img_peaks_1,76}
        \end{subfigure}
        \hfill
        \begin{subfigure}{0.5\linewidth} 
            \includegraphics[width=\linewidth]{images/peaks_1,89.png}
            \caption{}
            \label{img_peaks_1,89}
        \end{subfigure}
        \vfill
        \begin{subfigure}{0.5\linewidth} 
            \includegraphics[width=\linewidth]{images/peaks_2,03.png}
            \caption{}
            \label{img_peaks_2,03}
        \end{subfigure}
        \hfill
        \begin{subfigure}{0.5\linewidth} 
            \includegraphics[width=\linewidth]{images/peaks_2,23.png}
            \caption{}
            \label{img_peaks_2,23}
        \end{subfigure}
        \caption{Осциллограммы релаксационных колебаний в резонаторе.}
        \label{img_peaks}
    \end{figure}

    Выделим первые несколько пиков на осциллограмме в каждом случае. Среднее значение расстояния между пиками при заданном токе накачке -- период релаксационных колебаний $T$. Погрешность определения периода релаксационных колебаний $\sigma_T$ определяется как среднеквадратичное отклонение. Результаты этих вычислений представлены в таблице \ref{table_peaks_T_tau}.\par

    В этой же таблице приведены значения характерного времени затухания $\tau$, полученные из следующих соображений: поскольку колебания затухающие, интерполируем фронт затухающей экспонентой $U(t) = U_0 e^{-t/\tau}$. На рисунках \ref{img_peaks_1,19}-\ref{img_peaks_2,23} точки, по которым производилась интерполяция, отмечены красными маркерами. Логарифмируя, получим $lnU(t) = lnU_0 - \frac{t}{\tau}$. Тогда характерное время затухания $\tau$ может быть найдено как $\tau = -\frac{1}{k}$, где $k$ -- коэффициент наклона графика зависимости $t(lnU)$.

    \begin{table}[H]
    \centering
    \begin{tabular}{|c|c|c|c|c|c|c|}
    \hline
        № & 1 & 2 & 3 & 4 & 5 & 6 \\ \hline
        $I$, A & 1,19 & 1,49 & 1,76 & 1,89 & 2,03 & 2,23 \\ \hline
        $x$ & 1,31 & 1,64 & 1,93 & 2,08 & 2,23 & 2,45 \\ \hline
        $T$, мкс & 27,18 & 20,97 & 18,16 & 17,2 & 15,62 & 14,52 \\ \hline
        $\sigma_T$, мкс & 0,85 & 0,46 & 0,66 & 0,25 & 0,45 & 0,37 \\ \hline
        $\tau$, мкс & 241 & 162 & 220 & 213 & 115 & 158 \\ \hline
    \end{tabular}
    \caption{Результаты измерения $T$ и $\tau$ при различных значениях $I$.}
    \label{table_peaks_T_tau}
    \end{table}

    По данным таблицы \ref{table_peaks_T_tau} построим график зависимости частоты релаксационных колебаний $\nu = \frac{1}{T}$ от превышения над порогом $x = \frac{I}{I_{порог}}$. График представлен на рисунке \ref{img_nu(x)_compare_oscilloscope}.

    \begin{figure}[H]
        \centering
        \includegraphics[width=1\linewidth]{images/nu(x)_compare_oscilloscope.png}
        \caption{График зависимости $\nu(x)$.}
        \label{img_nu(x)_compare_oscilloscope}
    \end{figure}

    Из формулы (\ref{formula_omega(x)}):

    $$\frac{1}{\sqrt{\tau_{фотона}\tau_{уровня}}} = \frac{\omega}{\sqrt{x-1}} = 2\pi \frac{\nu}{\sqrt{x-1}} \equiv2\pi \cdot k, $$

    где $k$ -- коэффициент наклона графика на рисунке \ref{img_nu(x)_compare_oscilloscope}. Тогда

    $$\tau_{фотона} = \frac{1}{(2\pi k)^2 \tau_{уровня}} = \frac{1}{(2\pi \cdot 57,77 \cdot 10^3\ Гц)^2 \cdot (800 \cdot 10^{-6}\ c)} \approx 9\ нс$$

\subsection{Нахождение пиков релаксационного шума в спектре.}
    В режиме анализа спектра при различных значениях тока накачки $I$ будем находить пики релаксационного шума и измерять их положение $\nu$. Результаты измерений приведены в таблице \ref{table_nu_I_spectrscopy}.

    \begin{table}[H]
    \centering
    \begin{tabular}{|c|c|c|c|c|c|c|c|c|c|c|}
    \hline
        № & 1 & 2 & 3 & 4 & 5 & 6 & 7 & 8 & 9 & 10 \\ \hline
        $I$, А & 1,18 & 1,28 & 1,39 & 1,47 & 1,56 & 1,64 & 1,75 & 1,84 & 1,97 & 2,08 \\ \hline
        $x$ & 1,30 & 1,41 & 1,53 & 1,62 & 1,71 & 1,80 & 1,92 & 2,02 & 2,16 & 2,29 \\ \hline
        $\nu$, кГц & 22,25 & 27,86 & 31,52 & 33,72 & 37,14 & 38,85 & 42,02 & 45,7 & 49,35 & 51,54 \\ \hline
    \end{tabular}
    \caption{Результаты измерения частот пиков релаксационного шума при различных значениях $I$.}
    \label{table_nu_I_spectrscopy}
    \end{table}

    По данным таблицы \ref{table_nu_I_spectrscopy} построим график зависимости частоты релаксационных колебаний $\nu$ от превышения над порогом $x = \frac{I}{I_{порог}}$. График представлен на рисунке \ref{img_nu(x)_compare_spectroscopy}.

    \begin{figure}[H]
        \centering
        \includegraphics[width=1\linewidth]{images/nu(x)_compare_spectroscopy.png}
        \caption{График зависимости $\nu(x)$.}
        \label{img_nu(x)_compare_spectroscopy}
    \end{figure}

    Из формулы (\ref{formula_omega(x)}):

    $$\frac{1}{\sqrt{\tau_{фотона}\tau_{уровня}}} = \frac{\omega}{\sqrt{x-1}} = 2\pi \frac{\nu}{\sqrt{x-1}} \equiv2\pi \cdot k, $$

    где $k$ -- коэффициент наклона графика на рисунке \ref{img_nu(x)_compare_spectroscopy}. Тогда

    $$\tau_{фотона} = \frac{1}{(2\pi k)^2 \tau_{уровня}} = \frac{1}{(2\pi \cdot 44,3 \cdot 10^3\ Гц)^2 \cdot (800 \cdot 10^{-6}\ с)} \approx 16\ нс$$

\subsection{Определение частоты биения продольной моды.}
     Расстояние между двумя продольными модами в резонаторе волоконного лазера:
     
    $$\text{Условие синфазного сложения двух волн: } \Delta_{оптич} = \lambda \Rightarrow 2nL = \lambda = \frac{c}{\nu} \Rightarrow$$
    
    $$\Rightarrow \Delta \nu = \frac{c}{2nL} = 2\cdot10^7\ Гц = 20\ МГц,\ при\ n=\frac{3}{2}\ и \ L=5\ м.$$

    В режиме анализа спектра найдем частоту биения последней продольной моды $\Delta \nu$. На рисунке \ref{img_mode_peak_in_noise} данное значение отмечено зеленым маркером: $\Delta \nu = 20,03\ МГц$. По полученному значению уточним длину резонатора, полагая $n=\frac{3}{2}$:

    $$L_{эксп} = \frac{c}{2 n \Delta \nu} = 4,9925\ м = 4\ м\ 99\ см\ 25\ мм$$
    
    \begin{figure}[H]
        \begin{center}
        \includegraphics[width=0.55\linewidth]{images/mode_peak_in_noise.png}
        \end{center}
        \caption{Пик продольной моды на частоте 20 МГц.}
        \label{img_mode_peak_in_noise}
    \end{figure}

\section{Обсуждение результатов.}
\subsection{Измерение зависимости мощности лазера от тока накачки.}
    На рисунке \ref{img_two_P(P)} можно заметить различие в коэффициентах наклона и сдвиге по оси ординат двух участков линейной зависимости. Изменение происходит между 5 и 6 точками при $P_{накач} \sim 9$ Вт, т.е. при $I \sim 1,5$ А. Это связано с тем, что измеритель энергии лазерного излучения имеет три шкалы измерения, в том числе две шкалы $1,5 \div 15$ Дж и $15 \div 150$ В. Судя по всему, коэффициенты перевода $\kappa$ для этих двух шкал не равны. В нашем случае переключение шкал произошло как раз между 5 и 6 измерением.\par

    Согласно данным \textit{ru.wikipedia.org} КПД однополяризационного волоконного лазера может составлять порядка 25\%. Полученный нами $\eta =12,45\%$ не превышает этого значения.

\subsection{Исследование релаксационных колебаний.}
    Несовпадение точек на рисунке \ref{img_nu(x)_compare_oscilloscope} с ожидаемой зависимостью, описываемой формулой (\ref{formula_omega(x)}), возможно связано с некорректной работой осциллографа или нестационарным режимом генерации импульсов -- при наблюдении осциллограммы в течение более длительного времени наблюдались дефекты, подобные дефектам сигнала на рисунках \ref{img_peaks_2,03} и \ref{img_peaks_2,23}. \par

    Характерное время затухания релаксационных колебаний составляет $100\div 250$ мкс. Причины хаотичности зависимости $\tau(I)$, наблюдаемой в таблице \ref{table_peaks_T_tau}, установить не удалось.\par

    Теоретическое значение времени жизни фотона в резонаторе может быть оценено как:

    \begin{equation}
        \tau_{фотона} = - \frac{2L}{c\cdot ln(1-T)} \approx 150\ нс,
        \label{formula_tau_photon}
    \end{equation}

    где $L = 5$ м, $T = 0,2$ -- коэффициент пропускания зеркала резонатора. \par

    Данное значение не совпадает с полученным нами значением $\tau_{фотона} \approx 9$ нс. Возможно, это связано с неточностью оценки времени жизни уровня $\tau_{уровня}$: если бы $\tau_{уровня}$ было 80 мкс(в 10 раз меньше заявленного в описании к работе), то полученное нами значение совпадало бы с теоретическим значением по порядку величину.\par

    Отметим, что наблюдаемое число полных релаксационных колебаний(7) на рисунке \ref{img_peaks_1,19} согласуется с результатом $N_{колебаний} = 6$.

\subsection{Нахождение пиков релаксационного шума в спектре.}
    Точки на рисунке \ref{img_nu(x)_compare_spectroscopy} уже лучше описываются ожидаемой зависимостью, задаваемой формулой (\ref{formula_omega(x)}), чем на рисунке \ref{img_nu(x)_compare_oscilloscope}. Однако, совпадение все же не идеальное -- это может быть связано с низкой точностью проводимых измерений. На осциллографе в режиме анализа спектров наблюдается довольно широкий и зашумленный пик, что мешает его локализации. Тем не менее, характер зависимости все же просматривается.

    Теоретическое значение времени жизни фотона в резонаторе может быть оценено как:

    $$\tau_{фотона} = - \frac{2L}{c\cdot ln(1-T)} \approx 150\ нс,$$

    где $L = 5$ м, $T = 0,2$ -- коэффициент пропускания зеркала резонатора. \par

    Данное значение не совпадает с полученным нами значением $\tau_{фотона} \approx 16$ нс. Возможно, это связано с неточностью оценки времени жизни уровня $\tau_{уровня}$: если бы $\tau_{уровня}$ было 80 мкс(в 10 раз меньше заявленного в описании к работе), то полученное нами значение было бы очень близко к теоретическому.

\subsection{Определение частоты биения продольной моды.}
    Пик продольной моды едва различим среди шума на рисунке \ref{img_mode_peak_in_noise}. Его положение мало отличается от теоретического и дает пренебрежимо малую поправку в значение длины волокна. Поэтому данное различие следует считать случайным, учитывая соотношение амплитуды пика и амплитуды шумов.
    
\section{Заключение.}
    В работе исследованы релаксационные колебания в волоконном лазере при его ступенчатой накачке; получены зависимости мощности излучения волоконного лазера от мощности накачки и частоты релаксационных колебаний от превышения над порогом; вычислены значения пороговой мощности накачки, коэффициента полезного действия волоконного лазера и времени жизни фотонов в его резонаторе.
    
\section{Дополнительные вопросы и задачи.}
\subsection*{11. Температура и штарковское расщепление.}
    Пусть штарковское расщепление уровней рабочего перехода $\xi \sim 500\text{ см}^{-1}$. Тогда

    \begin{equation}
        E = \hbar \omega = \hbar c \frac{2 \pi}{\lambda} \sim k_{Б} T \Rightarrow T = \frac{2 \pi \hbar c}{k_{Б}} = \frac{h c}{\lambda k_{Б}} = \frac{h c}{k_{Б}} \xi = 720\ К
        \label{formula_T_Shtark}
    \end{equation}

\subsection*{12. Сравнение характеристик различных лазеров.}
    В таблице \ref{table_lasers_comparisson} приведены значения различных характеристик 4 лазеров. Время жизни фотона $\tau_{фотона}$ вычисляется по формуле (\ref{formula_tau_photon}), характерное время затухания $t_0$ -- по формуле (\ref{formula_t_0}), частота релаксационных колебаний $\omega$-- по формуле (\ref{formula_omega(x)}), число релаксационных колебаний до затухания $N_{колебаний} = \frac{t_0}{T}$. Вычисления проведены для значения превышения над порогом $x = 1,1$.

    \begin{table}[H]
    \begin{tabular}{|c|c|c|c|c|c|c|c|c|c|}
    \hline
        тип лазера & $L$, м & $\gamma$ & $\tau_{уровня}$ & $\tau_{фотона}$ & $t_0 $ & $\omega$, рад/с & $\nu$ & $T$  & $N_{колебаний}$  \\ \hline
        He-Ne & 0,5 & 0,99 & 100 нс & 15 нс & 182 нс & $8,2\cdot 10^6$ & 1,3 МГц & 768 нс  & \cellcolor[HTML]{FFCCC9}- \\ \hline
        YAG:Nd$^{3+}$ & 0,5 & 0,97 & 230 мкс & 15 нс & 418 мкс & $171\cdot 10^3$ & 27 кГц & 37 мкс & \cellcolor[HTML]{9AFF99}14 \\ \hline
        полупров. & 5\cdot 10^{-4} & 0,5 & 1 нс & 15 пс & 1,8 нс & 2,6 $10^9$ & 0,4 ГГц & 2,4 пс & \cellcolor[HTML]{FFCCC9}- \\ \hline
        волоконный & 5 & 0,12 & 800 мкс & 150 нс & 1,45 мс & $29\cdot 10^3$ & 4,6 кГц & 217 мкс & \cellcolor[HTML]{9AFF99}{6} \\ \hline
    \end{tabular}
    \caption{Сравнение характеристик различных лазеров.}
    \label{table_lasers_comparisson}
    \end{table}
     
\end{document}

    \begin{figure}[H]
        \centering
        \includegraphics[width=1\linewidth]{.png}
        \caption{.}
        \label{img_}
    \end{figure}