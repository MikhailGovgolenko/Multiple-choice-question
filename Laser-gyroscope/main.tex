% !TEX root = main.tex
\input{../tex-common/preamble.tex}

\title{Лазерный Гироскоп}
\author{Говголенко М.А.\\ Рождественский В.И.\\ Халезов Ф.А.}
\date{2025 г.}

\begin{document}
\maketitle

\section{Цель работы}
\begin{enumerate}[label=\textbullet]
    \item Исследовать принципы и теорию работы лазерного гироскопа.
    \item Провести измерения основных характеристик зеемановского лазерного гироскопа (частотной характеристики, масштабного коэффициента резонатора).
    \item Определить скорость вращения земли.
    \item Определить точность гироскопа.
\end{enumerate}


\section{Теория}
\subsection{Интерферометр Саньяка и кольцевой датчик}
В неподвижном интерферометре время обхода для обеих волн: $t=\frac{2\pi R}{c}$.

Пусть интерферометр вращается с угловой скорость $\Omega$, тогда точка встречи двух волн сместится:
\begin{equation}
    2 \pi R\pm R\Omega_{\pm}=ct_{\pm}\Rightarrow t_{\pm}=\frac{2\pi R}{c\mp R\Omega}
\end{equation}
где индекс «$+$» относится к волне, проходящей по контуру в направлении его вращения, «$-$» --- против направления вращения. Тогда разность времен обхода контура встречными волнами:
\begin{equation}
    \Delta t = t_{+}+t_{-}=2\pi R\left(\frac{1}{c-R\Omega}-\frac{1}{c+R\Omega}\right)=2\pi R\left(\frac{2R\Omega}{c^2-R^2\Omega^2}\right)\approx\frac{4\pi R^2\Omega}{c^2}.
\end{equation}
Следовательно, разность оптических путей:
\begin{equation}
    \Delta L=c\Delta t=\frac{4\pi R^2 \Omega}{c}.
    \label{eq:3}
\end{equation}

При $\pi R^2=\SI{1}{\metre^2}$ и $\Omega=2\pi\,\frac{\text{рад}}{\text{c}}$: $\Delta L\approx\SI{80}{\nano\metre}$.

Частота кругового лазера определяется как $\nu_m=m\frac cL ,m\in \mathbb{N}$, где $L$ --- длина резонатора.
Тогда для вращающегося лазера частоты генерации встречных волн:
\begin{equation}
    \nu_{\pm}=m\frac{c}{L\pm\frac{\Delta L}2}.
    \label{eq:4}
\end{equation}
Объединяя \eqref{eq:3} и \eqref{eq:4}:
\begin{equation}
    \Delta \nu=\nu_- -\nu_+ = mc\left(\frac{\Delta L}{L^2-\frac{\Delta L^2}{4}}\right)\approx\frac{mc}{L}\frac{\Delta L}{L}=\frac{\nu}{c}\frac{4\pi R^2\Omega}{cL}=\frac{4\pi R^2\Omega}{L\lambda}
    \label{eq:5}
\end{equation}

В плоскости NN наблюдаем интерференционную картину. Учитывая почти параллельность совмещаемых световых волн и малость разности их частот по сравнению с частотой генерации лазера, можно получить выражение для интенсивности света в точке х на плоскости NN:
\begin{equation}
    I=I_0\left[1+\cos\left(k\varepsilon x+\Delta\omega t+\varphi\right)\right]=I_0\left[1+\cos\left(\frac{4\pi n\theta x}{\lambda}+\Delta\omega t+\varphi\right)\right]
    \label{eq:6}
\end{equation}
где $\varphi$ --- некоторый постоянный сдвиг фаз, $\varepsilon=2n\theta$ --- угол расхождения волн после призмы, $\theta$ --- отклонение угла призмы от $90^\circ$, $\Delta\omega=2\pi\Delta\nu$.

Из \eqref{eq:6} видно, что при вращении гироскопа интерференционная картина бежит со скоростью, пропорциональной угловой скорости его вращения. Установив два фотоприемника на расстоянии равном ширине полосы интерференционной линии $\Lambda=\frac{\lambda}{\varepsilon}=\frac{\lambda}{2n\theta}$, можно подсчитать число интерференционных полос, прошедших в одном и другом направлениях, и определить угол поворота лазерного гироскопа. При вращении кольцевого лазера с постоянной угловой скоростью $\Omega$ число полос, пробегающих мимо фотоприемников за время $t$, равно
\begin{equation}
    N=\int\limits_0^t \Delta\nu dt=\frac{4\pi R^2}{\lambda L}\int\limits_0^t\Omega dt=\frac{4\pi R^2}{\lambda L}\gamma
    \label{eq:7}
\end{equation}
где $\gamma$ --- угол поворота гироскопа за время $t$. Отсюда видно, что один импульс на фотоприемнике соответствует углу поворота $\frac{\lambda L}{4\pi R^2}$ рад. Эту величину называют ценой импульса.

\begin{figure}
    \centering
    \includegraphics[width=0.5\linewidth]{images/image.png}
    \caption{Оптическая} схема смесителя ЛГ(детектора).
    \label{fig:1}
\end{figure}
\subsection{Устройство используемого лазерного микроскопа}
Схема исследуемого кольцевого четырёхзеркального лазерного гироскопа представлена на рис.~\ref{fig:2}. У этого гироскопа присутствуют несколько отличительных особенностей:
\begin{enumerate}
    \item Активная среда заполняет весь резонатор.
    \item Газоразрядные промежутки имеют общий катод.
    \item В резонаторе возбуждаются оптические колебания не с линейной, а с круговой поляризацией.
    \item Два из четырёх зеркал помещены на пьезокерамические элементы, с помощью которых
 длина резонатора может поддерживаться постоянной.
\end{enumerate}
\begin{figure}[H]
    \centering
    \includegraphics[width=0.5\linewidth]{images/image2.png}
    \caption{Оптическая схема четырёх зеркального гироскопа с зеемановской подставкой.}
    \label{fig:2}
\end{figure}

\begin{figure}[H]
    \centering
    \includegraphics[width=0.5\linewidth]{images/image3.png}
    \caption{Оптическая система непланарного резонатора.}
    \label{fig:3}
\end{figure}

\begin{figure}[H]
    \centering
    \includegraphics[width=0.5\linewidth]{images/image4.png}
    \caption{Экспериментальная установка.}
    \label{fig:4}
\end{figure}

\begin{figure}[H]
    \centering
    \includegraphics[width=0.5\linewidth]{images/image5.png}
    \caption{Расщепление контуров усиления активной среды в магнитном поле, приводящее к расщеплению частот встречных волн.}
    \label{fig:5}
\end{figure}

Зеемановский непланарный гироскоп имеет непланарный четырёхзеркальный резонатор (рис.~\ref{fig:2}), в котором одно зеркало сферическое, а три других плоские. Можно считать, что оптический контур симметричный, то есть все стороны плеч резонатора равны. Углы падения на зеркала и углы между плоскостями падения от одного зеркала к другому также одинаковые и определяются углом излома резонатора $\beta$.

Использование непланарного резонатора приводит к повороту при каждом отражении от зеркала поперечного распределения поля моды на угол, соответствующий углу между плоскостями падения на соседние зеркала, и в результате к расщеплению частот для мод с ортогональными поляризациями. Расщепление частоты для мод с левой и правой круговой поляризацией оказывается равным: $\Delta\nu_{rt}=\frac{c}{L}\frac{\rho}{\pi}$, где $\rho = \sum\limits_i \varphi_i$ --- суммарный угол поворота поперечного распределения моды за обход резонатора, а $\varphi_i$– поворо распределения поля на каждом зеркале. В зеемановском двухчастотном гироскопе $\varphi_i=\frac{\pi}{8}$, $\rho=\frac{\pi}{2}$, $L=\SI{0.2}{\metre}$, что даёт сдвиг частот $\Delta\nu_{rl}=\SI{749.48}{\mega\hertz}$, что есть половина интервала между продольными модами.

Благодаря круговым поляризациям волн, лазер очень чувствителен к магнитному полю, поскольку при наложении магнитного поля параллельно оптической оси резонатора контур
усиления активной среды расщепляется надвое --- для волн с встречным вращением вектора поляризации (рис.~\ref{fig:5}). Из-за эффекта затягивания частоты каждая из встречных волн затягивается к центру своего контура усиления. В результате этого можно искусственно создавать разность частот встречных волн, что позволяет имитировать вращение гироскопа.
\section{Ход работы}
\subsection{Измерение скорости вращения Земли}
Для определения скорости вращения Земли проведены четыре измерения угловой скорости гироскопа: для двух направлений оси чувствительности для двух ортогональных мод. Итоговое значение вычислим как полуразность средних для двух направлений оси чувствительности (среднее считается для одинкового направления оси):
\begin{center}
    \begin{tabular}{|c|c|c|c|c|}
\hline
N & 1 & 2 & 3 & 4 \\
\hline
$\Omega,\,^{\circ}/\text{ч}$ & --1.436 & 0.975 & 11.116 & 25.399 \\
\hline
\end{tabular}
\end{center}
\[\Omega=\frac{\frac{\Omega_1+\Omega_2}{2}-\frac{\Omega_3+\Omega_4}{2}}{2}=9.224\,^\circ/\text{ч}\].

Сравним полученное значение с истинным. Угловая частотам вращения Земли вокруг своей оси $\Omega_{\text{з}}=15\,^\circ/\text{ч}$. на широте $55^\circ$ она равна
\[\Omega_{\text{з}}'=\Omega_{\text{з}}\cdot \sin 55^\circ\approx 12.28\,^\circ/\text{ч}\].

Присутствует значительное для лазерного гироскопа расхождение с теоретически расчитанным результатом. Это может быть с неисправностью самого гиросокопа или ошибкой накопления данных.


\subsection{Измерение скорости вращения Земли волоконным гироскопом}
Для определения скорости вращения Земли волоконным гироскопом проведем два измерения
для двух направлений оси чувствительности и итоговое значение вычислим как их полураз-
ность:
\[\Omega=\frac{\Omega_1-\Omega_2}{2}=11.2841\,^{\circ}/\text{ч},\]
где $\Omega_1=25.042\,^{\circ}/\text{ч}$, $\Omega_1=2.4738\,^{\circ}/\text{ч}$. Ошибка измерения скорости вращения Земли составила
$8\%$.
\subsection{Измерение выходной характеристики}
\begin{figure}[H]
    \centering
    \begin{tikzpicture}
        \begin{axis}[
            xlabel={Ток $I$, мА},
            ylabel={Частота $\Delta \nu$, Гц},
            grid=both,
            width=0.5\textwidth,
            height=0.5\textwidth,
            xmin=-15, xmax=15,
            ymin=-3000, ymax=3000,
            xtick distance=5,
            ytick distance=1000,
            legend pos=north west,
        ]
        % Экспериментальные точки
        \addplot[
            only marks,
            mark=*,
            color=blue,
        ] coordinates {
            (14.01, 2330)
            (13.38, 2216)
            (11.94, 1970)
            (10.06, 1630)
            (8.33, 1300)
            (6.82, 980)
            (5.57, 702)
            (5.03, 570)
            (-14.64, -2400)
            (-12.48, -2000)
            (-9.17, -1400)
            (-8.77, -1300)
            (-7.44, -1000)
            (-6.66, -854)
            (-5.59, -540)
            (-4.19, 0)
        };

        \addplot[
            domain=-15:15,
            samples=2,
            color=red,
            thick,
        ] {152.32*x-28.17};
        
        \addlegendentry{$I_a=\SI{1.2}{\milli\ampere}$}

        % Стрелка сверху к центру
        \draw[->, dashed, thick] (axis cs:10,2500) -- (axis cs:5,1500);
        \node at (axis cs:5,2100) {1}; % подпись к стрелке сверху

        % Стрелка снизу к центру
        \draw[->, dashed, thick] (axis cs:-10,-2500) -- (axis cs:-5,-1500);
        \node at (axis cs:-5,-2100) {2}; % подпись к стрелке снизу
        
    \end{axis}
\end{tikzpicture}

\caption{График зависимости частоты $\Delta\nu$ от тока $I$ при токе накачки $I_a=\SI{1.2}{\milli\ampere}$.}
\label{fig:6}
\end{figure}

\begin{figure}[H]
    \centering
    \begin{tikzpicture}
        \begin{axis}[
            xlabel={Ток $I$, мА},
            ylabel={Частота $\Delta \nu$, Гц},
            grid=both,
            width=0.5\textwidth,
            height=0.5\textwidth,
            xmin=-15, xmax=15,
            ymin=-3000, ymax=3000,
            xtick distance=5,
            ytick distance=1000,
            legend pos=north west,
        ]
        % Экспериментальные точки
        \addplot[
            only marks,
            mark=*,
            color=blue,
        ] coordinates {
            (13.98, 2250)
            (8.91, 1350)
            (8.18, 1200)
            (7.34, 1010)
            (6.43, 830)
            (5.61, 640)
            (-5.33, -590)
            (-7.11, -970)
            (-8.65, -1300)
            (-9.2, -1380)
            (-9.56, -1480)
            (-11.84, -1870)
            (-13.11, -2100)
            (-14.64, -2400)
        };

        % Аппроксимационная прямая
        \addplot[
            domain=-15:15,
            samples=2,
            color=red,
            thick,
        ] {152.32*x-28.17}; % коэффициенты приблизительно по МНК
        \addlegendentry{$I_a=\SI{1.2}{\milli\ampere}$}

        % Стрелка сверху к центру
        \draw[->, dashed, thick] (axis cs:10,2500) -- (axis cs:5,1500);
        \node at (axis cs:5,2100) {1}; % подпись к стрелке сверху

        % Стрелка снизу к центру
        \draw[->, dashed, thick] (axis cs:-5,-1500) -- (axis cs:-10,-2500);
        \node at (axis cs:-5,-2100) {2}; % подпись к стрелке снизу
        
    \end{axis}
\end{tikzpicture}

\caption{График зависимости частоты $\Delta\nu$ от тока $I$ при токе накачки $I_a=\SI{1.2}{\milli\ampere}$. Красная линия — аппроксимация методом наименьших квадратов.}
\label{fig:7}
\end{figure}

\begin{figure}[H]
    \centering
    \begin{tikzpicture}
        \begin{axis}[
            xlabel={Ток $I$, мА},
            ylabel={Частота $\Delta \nu$, Гц},
            grid=both,
            width=0.5\textwidth,
            height=0.5\textwidth,
            xmin=-15, xmax=15,
            ymin=-3000, ymax=3000,
            xtick distance=5,
            ytick distance=1000,
            legend pos=north west,
        ]
        % Экспериментальные точки
        \addplot[
            only marks,
            mark=*,
            color=blue,
        ] coordinates {
            (-14.67, -2470)
            (-11.89, -1971)
            (-10.46, -1710)
            (-8.60, -1400)
            (-7.62, -1200)
            (-6.75, -1000)
            (-6.37, -925)
            (-5.15, -660)
            (-3, 0)
            (5.54, 780)
            (6.29, 940)
            (7.11, 1100)
            (8.18, 1300)
            (10.4, 1700)
            (12.85, 2100)
            (13.99, 2330)
        };

        \addplot[
            domain=-15:15,
            samples=2,
            color=red,
            thick,
        ] {160.62*x+34.018};
        
        \addlegendentry{$I_a=\SI{1.3}{\milli\ampere}$}

        % Стрелка сверху к центру
        \draw[->, dashed, thick] (axis cs:5,1500) -- (axis cs:10,2500);
        \node at (axis cs:5,2100) {2}; % подпись к стрелке сверху

        % Стрелка снизу к центру
        \draw[->, dashed, thick] (axis cs:-10,-2500) -- (axis cs:-5,-1500);
        \node at (axis cs:-5,-2100) {1}; % подпись к стрелке снизу
        
    \end{axis}
\end{tikzpicture}

\caption{График зависимости частоты $\Delta\nu$ от тока $I$ при токе накачки $I_a=\SI{1.3}{\milli\ampere}$.}
\label{fig:8}
\end{figure}

\begin{figure}[H]
    \centering
    \begin{tikzpicture}
        \begin{axis}[
            xlabel={Ток $I$, мА},
            ylabel={Частота $\Delta \nu$, Гц},
            grid=both,
            width=0.5\textwidth,
            height=0.5\textwidth,
            xmin=-15, xmax=15,
            ymin=-3000, ymax=3000,
            xtick distance=5,
            ytick distance=1000,
            legend pos=north west,
        ]
        % Экспериментальные точки
        \addplot[
            only marks,
            mark=*,
            color=blue,
        ] coordinates {
            (11.20, 1840)
            (10.49, 1707)
            (9.18, 1457)
            (7.88, 1220)
            (6.02, 846)
            (4.7, 527)
            (3.42, 0)
            (-5.42, -580)
            (-6.60, -890)
            (-8.33, -1270)
            (-10.09, -1578)
            (-12.58, -2000)
            (-14.65, -2400)
        };

        \addplot[
            domain=-15:15,
            samples=2,
            color=red,
            thick,
        ] {160.62*x+34.018};

        \addlegendentry{$I_a=\SI{1.3}{\milli\ampere}$}

        % Стрелка сверху к центру
        \draw[->, dashed, thick] (axis cs:10,2500) -- (axis cs:5,1500);
        \node at (axis cs:5,2100) {1}; % подпись к стрелке сверху

        % Стрелка снизу к центру
        \draw[->, dashed, thick] (axis cs:-5,-1500) -- (axis cs:-10,-2500);
        \node at (axis cs:-5,-2100) {2}; % подпись к стрелке снизу
        
    \end{axis}
\end{tikzpicture}

\caption{График зависимости частоты $\Delta\nu$ от тока $I$ при токе накачки $I_a=\SI{1.3}{\milli\ampere}$.}
\label{fig:9}
\end{figure}

\begin{figure}[H]
    \centering
    \begin{tikzpicture}
        \begin{axis}[
            xlabel={Ток $I$, мА},
            ylabel={Частота $\Delta \nu$, Гц},
            grid=both,
            width=0.5\textwidth,
            height=0.5\textwidth,
            xmin=-15, xmax=15,
            ymin=-3000, ymax=3000,
            xtick distance=5,
            ytick distance=1000,
            legend pos=north west,
        ]
        % Экспериментальные точки
        \addplot[
            only marks,
            mark=*,
            color=blue,
        ] coordinates {
            (-14.66, -2340)
            (-11.15, -1680)
            (-8.95, -1223)
            (-7.24, -851)
            (-6.48, -675)
            (-5.42, -385)
            (-3.86, 0)
            (13.98, 2203)
            (10.39, 1530)
            (9.29, 1287)
            (8.22, 1070)
            (7.00, 820)
            (5.82, 387)
            (4.27, 0)
        };

        \addplot[
            domain=-15:15,
            samples=2,
            color=red,
            thick,
        ] {160.62*x+34.018};
        
        \addlegendentry{$I_a=\SI{1.3}{\milli\ampere}$}

        % Стрелка сверху к центру
        \draw[->, dashed, thick] (axis cs:10,2500) -- (axis cs:5,1500);
        \node at (axis cs:5,2100) {2}; % подпись к стрелке сверху

        % Стрелка снизу к центру
        \draw[->, dashed, thick] (axis cs:-10,-2500) -- (axis cs:-5,-1500);
        \node at (axis cs:-5,-2100) {1}; % подпись к стрелке снизу
        
    \end{axis}
\end{tikzpicture}

\caption{График зависимости частоты $\Delta\nu$ от тока $I$ при токе накачки $I_a=\SI{1.3}{\milli\ampere}$.}
\label{fig:10}
\end{figure}

\begin{figure}[H]
    \centering
    \begin{tikzpicture}
        \begin{axis}[
            xlabel={Ток $I$, мА},
            ylabel={Частота $\Delta \nu$, Гц},
            grid=both,
            width=0.5\textwidth,
            height=0.5\textwidth,
            xmin=-15, xmax=15,
            ymin=-3000, ymax=3000,
            xtick distance=5,
            ytick distance=1000,
            legend pos=north west,
        ]
        % Экспериментальные точки
        \addplot[
            only marks,
            mark=*,
            color=blue,
        ] coordinates {
            (-14.65, -2400)
            (-11.08, -1800)
            (-8.57, -1300)
            (-6.93, -1000)
            (-5.91, -782)
            (-4.83, -555)
            (-3.10, 0)
            (14.04, 2270)
            (10.36, 1620)
            (7.89, 1180)
            (6.49, 945)
            (5.13, 660)
            (3.6, 200)
            (2.8, 0)
        };

        \addplot[
            domain=-15:15,
            samples=2,
            color=red,
            thick,
        ] {151.22*x-17.3};
        
        \addlegendentry{$I_a=\SI{1.4}{\milli\ampere}$}

        % Стрелка сверху к центру
        \draw[->, dashed, thick] (axis cs:10,2500) -- (axis cs:5,1500);
        \node at (axis cs:5,2100) {2}; % подпись к стрелке сверху

        % Стрелка снизу к центру
        \draw[->, dashed, thick] (axis cs:-10,-2500) -- (axis cs:-5,-1500);
        \node at (axis cs:-5,-2100) {1}; % подпись к стрелке снизу
        
    \end{axis}
\end{tikzpicture}

\caption{График зависимости частоты $\Delta\nu$ от тока $I$ при токе накачки $I_a=\SI{1.4}{\milli\ampere}$.}
\label{fig:11}
\end{figure}

\begin{figure}[H]
    \centering
    \begin{tikzpicture}
        \begin{axis}[
            xlabel={Ток $I$, мА},
            ylabel={Частота $\Delta \nu$, Гц},
            grid=both,
            width=0.5\textwidth,
            height=0.5\textwidth,
            xmin=-15, xmax=15,
            ymin=-3000, ymax=3000,
            xtick distance=5,
            ytick distance=1000,
            legend pos=north west,
        ]
        % Экспериментальные точки
        \addplot[
            only marks,
            mark=*,
            color=blue,
        ] coordinates {
            (-14.67, -2300)
            (-11.12, -1711)
            (-8.83, -1300)
            (-7.14, -1036)
            (-6.07, -823)
            (-4.83, -570)
            (-3.10, 100)
            (-3.08, 0)
            (14.06, 2100)
            (10.75, 1630)
            (8.02, 1200)
            (5.35, 675)
            (4.5, 470)
            (3.01, 0)
        };

        \addplot[
            domain=-15:15,
            samples=2,
            color=red,
            thick,
        ] {144.31*x+23.76};

        \addlegendentry{$I_a=\SI{1.5}{\milli\ampere}$}

        % Стрелка сверху к центру
        \draw[->, dashed, thick] (axis cs:10,2500) -- (axis cs:5,1500);
        \node at (axis cs:5,2100) {2}; % подпись к стрелке сверху

        % Стрелка снизу к центру
        \draw[->, dashed, thick] (axis cs:-10,-2500) -- (axis cs:-5,-1500);
        \node at (axis cs:-5,-2100) {1}; % подпись к стрелке снизу
        
    \end{axis}
\end{tikzpicture}
\caption{График зависимости частоты $\Delta\nu$ от тока $I$ 
\label{fig:12}
при токе накачки $I_a=\SI{1.5}{\milli\ampere}$.}
\end{figure}

Рассчитаем ширину зоны захвата $\Delta$ для каждой серии измерений. На рисунках \ref{fig:6}--\ref{fig:12} можно измерить ширину зоны захвата $\tilde{\Delta}$, выраженную в мА. Для перевода этих значений в Гц, найдем коэффициенты перевода $K$ как коэффициенты наклона линейной (удаленной от зоны захвата) части графиков выходной характеристики. Тогда верна формула $\Delta=K\cdot\tilde{\Delta}$, где $\Delta$ --- значение ширины зоны захвата в Гц. Полученные значения занесем в таблицу.

\begin{table}[H]
    \centering
    \begin{tabular}{|c|c|c|c|c|}
    \hline
    $I_a$, мА & 1.2 & 1.3 & 1.4 & 1.5 \\
    \hline
    $\tilde{\Delta}$, мА & 8.38 & 8.13 & 5.90 & 6.09 \\
    \hline
    $K$, Гц/мА & 152.32 & 160.62 & 151.22 & 144.31 \\
    \hline
    $\Delta$, Гц & 1276.44 & 1305.84 & 892.20 & 878.85 \\
    \hline
    \end{tabular}
    \caption{Параметры гироскопа при различных токах накачки}
    \label{tab:1}
\end{table}



Отобразим зависимость $\Delta$ от $I_a$:

\begin{figure}[H]
    \centering
    \begin{tikzpicture}
        \begin{axis}[
            xlabel={$I_a$, мА},
            ylabel={$\Delta$, Гц},
            grid=both,
            width=0.5\textwidth,
            height=0.4\textwidth,
            ymin=0, ymax=1500,
            xmin=1.1, xmax=1.6,
            xtick={1.2,1.3,1.4,1.5},
            ytick distance=200,
            legend pos=north east,
        ]
        % Точки из таблицы
        \addplot[
            only marks,
            mark=*,
            color=blue,
        ] coordinates {
            (1.2, 1276.44)
            (1.3, 1305.84)
            (1.4, 892.20)
            (1.5, 878.85)
        };


        % Аппроксимация прямой (по желанию)
        \addplot[
            domain=1.2:1.5,
            samples=2,
            color=red,
            thick,
        ] {-1606.4*x+3257};

        \end{axis}
    \end{tikzpicture}
    \caption{График зависимости ширины зоны захвата $\Delta$ от тока накачки $I_a$.}
    \label{fig:delta_vs_Ia}
\end{figure}


\section{Выводы}
В лабораторной работе мы ознакомились с принципами устройства и исследовали работу зеемановского лазерного гироскопа.

Получены частотные характеристики при различных токах накачки (см. рис.~\ref{fig:6}--\ref{fig:12}) и определены его основные параметры (см. таб.~\ref{tab:1}). Измерена скорость вращения Земли.

Частотные характеристики хорошо аппроксимируются прямой на большом расстоянии от области захвата частот. Зависимость ширины зоны захвата $\Delta$ от тока накачки $I_a$ получилась тоже линейной.



\section{Контрольные вопросы и упражнения}

\textcolor{cyan}{Задачи, которые нужно было исправить, выделены цветом.}
\vspace{1em}


\textbf{1.} В чём недостатки пассивного интерферометра Саньяка? Какова должна быть его площадь, чтобы можно было измерить угловую скорость вращения Земли? Длина волны кольцевого лазера 633 нм, разрешение интерферометра по разности фаз $\frac{\pi}{4}$.

\textbf{Решение.} По формуле \eqref{eq:3}:
\[\Delta L=\frac{4S\Omega}{c}=\frac{\lambda \Delta\varphi}{2\pi}=\frac{\lambda}{8}.\]
Угловая скорость земли:
\[\Omega=\frac{2\pi}{T}\approx 7.2921150\cdot10^{-5}\,\text{Гц}.\]
Тогда
\[S=\frac{c\lambda}{32\Omega}\approx8.13\cdot10^{4}\,\text{м}^2.\]
Одним из существенных недостатков получается большая площадь контура.
\vspace{1em}

\textbf{2.} Вывести выражение \eqref{eq:6}.

\textbf{Решение.} Поле в месте встречи волн равно $\vec{E}=\vec{E}_1+\vec{E}_2$. Тогда
\[
E^2 = E_1^2 + E_2^2 - 2E_1E_2\cos\Delta\varphi,
\qquad
I \sim E^2,
\qquad
I_1 \approx I_2,
\qquad
I \sim I_0(1+\cos\Delta\varphi).
\]
Где $\Delta\varphi$ --- разность фаз.
Так как $\frac{\Delta \lambda}{\lambda}=\frac{\Delta\varphi_0}{2\pi}=\frac{2n\theta x}{\lambda}$, то $\Delta\varphi_0=\frac{2\pi\varepsilon}{\lambda}$.
Исходная разность фаз $\Delta\varphi$ равна сумме $\Delta\varphi_0$, начальной разности фаз $\phi$ и набега фазы $\Delta\omega t$ в связи с разностью частот.
\[\Delta\varphi=\frac{2\pi\varepsilon}{\lambda}+\Delta\omega t+\phi.\]
В итоге получаем
\[I = I_0\left(1+\cos\left(\frac{2\pi\varepsilon}{\lambda}+\Delta\omega t+\phi\right)\right).\]
\vspace{1em}

\textbf{3.} Почему настройка на максимум кривой усиления производится по амплитуде сигнала биений и максимуму частоты биений?

\textbf{Ответ.} Биения предсавляются в виде
\[x(t)=2x_0\cos\left(\frac{\omega_1-\omega_2}{2}t\right)\cos\left(\frac{\omega_1+\omega_2}{2}t\right).\]
Тогда если мы определим её максимальную частоту, она будет являться разностью между частотами встречных волн.

В зеемановском резонаторе встречные волны поляризованы ортогонально (с круговой поляризацией), вследствие этого из-за эффекта затягивания частот каждая из встречных волн затягивается к центру своего контура усиления. Зная разность этих частот, как было выведено выше, можно рассчитать скорость вращения гироскопа по формуле \eqref{eq:5}.
\vspace{1em}

\textcolor{cyan}{\textbf{4.} Какой из эффектов, определяющий погрешности лазерного гироскопа, представляет наибольшую проблему?}

\textbf{Ответ.} Наибольшую проблему представляет сдвиг нуля. Это связано с тем, что даже при отсутствии вращения гироскопа возникает некоторая разность частот встречных волн, которая может быть сопоставима с частотой биений при измерении малых угловых скоростей. Сдвиг нуля возникает из-за асимметрии амплитуд встречных волн, вызванной неоднородностями активной среды, а также из-за влияния внешних факторов, таких как вибрации и температурные изменения.
\vspace{1em}

\textbf{5.} Почему нельзя измерить угловую скорость по разности фаз встречных волн внутри
 зоны захвата?

\textbf{Ответ.} При угловой скорости, меньшей пороговой, т.е. внутри зоны захвата, встречные волны будут иметь одинаковую частоту и между ними установится постоянная разность фаз, которая не зависит от скорости вращения. То есть по разности фаз уже не определить угловую скорость гироскопа.
\vspace{1em}

\textbf{6.} В чем физическая сущность эффекта Зеемана?

\textbf{Ответ.} Эффект Зеемана --- это расщепление энергетических уровней атома в магнитном поле, возникающее из-за взаимодействия магнитного момента атома с полем. Энергия подуровней изменяется на величину $\Delta E=-\vec{\mu}\vec{B}$ и поэтому спектральная линия делится на несколько компонент.
\vspace{1em}

\textbf{7.} Вычилите масштабный коэффициент для кольцевого гелий-неонового лазера с длиной резонатора $20\, \text{см}$ и площадью контура $25\, \text{см}^2$ и сравните результат с экспериментальным значением.

\textbf{Решение.}
\[K=\frac{4S}{\lambda L}\approx \SI{78369.906}{\hertz},\]
где $\lambda=\SI{638}{\nano\metre}$.
\vspace{1em}

\textbf{8.} Оцените зону захвата для гироскопа с параметрами, указанными в вопросе 7. Коэффициент рассеяния зеркал $R_S = 0.3\%$. Амплитуды встречных волн считать одинаковыми. Возможно ли измерить угловую скорость вращения Земли при помощи такого гироскопа, не используя частотные подставки? \textcolor{cyan}{найти циклическую частоту (разность частот встречных волн) входа в захват.}

\textbf{Решение.} Используем формлы (35), (36) и (37) из лабораторного практикума.

\begin{equation*}
    R\approx R_s \frac{\lambda}{2\pi \omega_0},
\end{equation*}

\begin{equation*}
    \Delta\nu=K\Omega;\quad K=\frac{4S}{\lambda L},
\end{equation*}

\begin{equation*}
    \Omega_L'=r_2\frac{E_1}{E_2}\frac{c}{L}.
\end{equation*}
$R_s=4\cdot3\cdot10^{-3}$ так как зеркал резонатора 4. Считаем, что $\omega_0=\SI{0.53}{\milli\metre},\,\frac{E_1}{E_2}\approx1,\, r_2=R$. Получаем зону захвата:
\[\Omega_L' = \frac{R_S \lambda}{2\pi \omega_0} \frac{c K \lambda}{4S}
\approx \SI{3448.6}{\hertz} \; \textcolor{cyan}{\approx 2.2 \cdot 10^4 \frac{\text{рад}}{\text{с}}}.\]
Но частота биений гироскопа равна $\Delta\nu=K\Omega_\text{з}\approx\SI{5.71}{\hertz}$. А значит происходит захват частоты. То есть измерить угловую скорость вращения земли не удастся.
\vspace{1em}

\textbf{9.} Каким образом при помещении активной среды зеемановского гироскопа в магнитное поле происходит расщепление частот встречных волн?

\textbf{Ответ.} 
Благодаря эффекту Зеемана создаётся два контура усиления, смещенных по частоте. Благодаря эффекту затягивания частоты генерации, каждая встречная волна смещается к центру своего контура, что создает расщепление частот встречных волн.
\vspace{1em}

\textbf{10.} В чем отличия механического гироскопа от лазерного и от интерферометра Саньяка?

\textbf{Ответ.} В механическом гироскопе вращается некоторое тело (ротор). За счёт закона сохранения углового момента ротор сопротивляется изменению направления оси вращения. При помощи него измеряется угол отклонения оси ротора относительно корпуса. 

В лазерном гирскопе используется кольцевой лазер, в котором возникают две встречные волны. При вращении контура лазера возникает эффект Саньяка: частоты волн различаются пропорционально угловой скорости вращения.
При помощи него измеряется угловая скорость вращения корпуса, на котором он установлен.

В интерферометре Саньяка используется оптоволокно: лазерный свет направляется по длинной витой петле, образуя два встречных пучка. При вращении контура пучки накладываются сдвигом фаз (эффект Саньяка), создавая интерференционную картину. При помощи него измеряется угловая скорость вращения корпуса, на котором он установлен.
\vspace{1em}

\textbf{11.} Каковы аналоги явления захвата частоты в механике и электричестве?

\textbf{Ответ.} Явление захвата частоты является примером общего явления синхронизации колебании в колебательной системе с нелинейной обратной связью. Механическим аналогом этого явления является, например, установление общей частоты для каждого связанного маятника в системе, в которой изначально все маятники осцилировали на близких частотах. Электричестким аналогом является, например, электронные системы с фазовой автоподстройкой где осциллятор подстраивается под фазу внешнего сигнала, частота синхронизируется, разность фаз стабилизируется.
\vspace{1em}

\textbf{12.} В чем сущность явления затягивания частоты в лазере?

\textbf{Ответ.} Явление затягивания частоты связано с аномальной дисперсией активной среды вблизи линии усиления.

Показатель преломления $n$ зависит от частоты: для частот ниже центра линии \mbox{$(\nu<\nu_0)$} он увеличивается, а для частот выше --- уменьшается.

Поскольку собственные частоты резонатора определяются как $\nu_m=\frac{mc}{2nL}$ то при увеличении $n$ частоты сдвигаются вниз, а при уменьшении $n$ --- вверх.
Таким образом, частоты всех мод смещаются в сторону центра линии усиления, что и называется затягиванием частоты генерации.
\vspace{1em}

\textbf{13.} Рассчитайте разность хода встречных волн для вращающегося со скоростью Земли лазерного гироскопа с параметрами, указанными в вопросе 7.

\textbf{Решение.} Из формулы \eqref{eq:3} получаем $\Delta L=\frac{4S\Omega_\text{з}}{c}\approx\SI{2.4}{\femto\metre}$.
\vspace{1em}

\textbf{14.}  Определите величину частотной подставки, создаваемой эффектом Зеемана, если ширина контура усиления $\Delta ν_0=\SI{1.5}{\giga\hertz}$, продольное магнитное поле $H=7.5\,\text{Э}$, пропускание выходного зеркала $T = 0.05\%$, периметр резонатора $L =\SI{20}{\centi\metre}$, $G_0/k \approx 2$.

\textbf{Решение.} По формуле (33) из лабораторного практикума:
\[\Delta\nu\approx2\frac{G_0}{k}\frac{\Delta\nu_z\Delta\nu_r}{\Delta\nu_0}.\]
Используя формулы (30) $\left(\Delta\nu_z=\pm\frac{g\mu_B}{h}H~(\text{СИ})\right)$ и (32) $\left(\Delta\nu_r=\frac{\gamma c}{2\pi L},~\text{где}~ \gamma\approx T\right)$ лабораторного практикума, получим:
\[\Delta\nu=2\frac{G_0}{k}\frac{g\mu_B}{h}H\frac{Tc}{2\pi L}\approx\SI{4.34}{\kilo\hertz}.\]
\vspace{1em}

\textbf{15.} Рассчитать, как изменится чувствительность (величина масштабного коэффициента) лазерного гироскопа в том случае, если в нем использовать волоконный кольцевой резонатор в виде волокна, намотанного на катушку, в котором в начале был один виток, а потом $N$ витков?

\textbf{Решение.} Если для одного витка $K_1=\frac{4S}{L\lambda}$, то для $N$ витков площадь и длина контура увеличатся в $N$ раз соответсвенно. А значит \[K_N=\frac{4NS}{NL\lambda}=K_1,\] то есть чувстительность не изменится.

Приветствую! Спасибо за использование моих услуг. Если у вас есть ещё вопросы или нужна помощь, не стесняйтесь обращаться. Удачи в ваших проектах и исследованиях!

\end{document}
